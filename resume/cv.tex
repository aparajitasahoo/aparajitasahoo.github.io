\documentclass[margin,line]{res}

\usepackage{hyperref}
\usepackage{amsmath}
\usepackage{textcomp}
\usepackage{color}
\usepackage{lettrine}
%\usepackage[left=0.75in, right=0.75in, top=0.75in,bottom=0.75in, margin=0.75in]{geometry}
\oddsidemargin = -.5in
\evensidemargin = -.5in
\topmargin = -0.3in
\textheight = 9.75in

\textwidth = 6.0in
\itemsep=0in
\parsep=0in
% if using pdflatex:
%\setlength{\pdfpagewidth}{\paperwidth}
%\setlength{\pdfpageheight}{\paperheight} 

\newenvironment{list1}{
  \begin{list}{\ding{113}}{%
      \setlength{\itemsep}{0in}
      \setlength{\parsep}{0in} \setlength{\parskip}{0in}
      \setlength{\topsep}{0in} \setlength{\partopsep}{0in} 
      \setlength{\leftmargin}{0.17in}}}{\end{list}}
\newenvironment{list2}{
  \begin{list}{$\bullet$}{%
      \setlength{\itemsep}{0in}
      \setlength{\parsep}{0in} \setlength{\parskip}{0in}
      \setlength{\topsep}{0in} \setlength{\partopsep}{0in} 
      \setlength{\leftmargin}{0.2in}}}{\end{list}}

\tolerance=1
\emergencystretch=\maxdimen
\hyphenpenalty=10000
\hbadness=10000

\pagenumbering{gobble}

\begin{document}

%\name{\textbf{\huge{Alankar Kotwal}} \vspace*{.05in} \newline  {\sc Senior Undergraduate}\vspace*{.1in}}
\name{\textbf{\huge{Aparajita Sahoo}} \vspace*{.1in} }

\begin{resume}
\section{\sc Contact Information}
\vspace{.05in}
\begin{tabular}{@{}p{2.9in}p{.5in}p{3in}}
\href{https://redxmumbai.com/}{\textcolor{blue} {REDX Innovation Lab, Mumbai}} & \multicolumn{1}{r}{\it Phone:}  &(+91) 907 621 1711 \\           
\href{http://redx.io/}{\textcolor{blue} {Rethinking Engineering Design eXecution}} &\multicolumn{1}{r}{\it E--Mail:}& \href{mailto:aparajitasahoois@gmail.com}{\textcolor{blue}{aparajitasahoois@gmail.com}} \\ 
Welingkar Building, Lakhamsi Nappu Road & & \href{mailto:aparajita.redx@gmail.com}{\textcolor{blue}{aparajita.redx@gmail.com}} \\ 
Matunga (East), Mumbai, India 400 019 & \multicolumn{1}{r}{\it Webpage:} &\href{http://aparajitasahoo.github.io/}{\textcolor{blue}{aparajitasahoo.github.io}} \\     
\end{tabular}

\section{\sc Research Interests}
\lettrine[lines=2]{I}{} am passionate about creating a social impact by working at the intersection of health technologies, data science, design and social innovation that will impact the lives of the people globally. I am also interested in enhancing  user experiences by combining the power of computing, mobile technologies and effective design. I have also worked on electronics prototyping of low cost healthcare devices that can be deployed in rural areas that don’t have access to medical facilities.  

\section{\sc Education}
{\bf \href{http://mu.ac.in/portal/}{\textcolor{blue}{Univ. of Mumbai}}, \href{http://fcrit.ac.in/}{\textcolor{blue}{Fr. Conceicao Rodrigues Inst. of Technology}}}, Mumbai \hfill {\it 2011 -- 2015} \\
\vspace*{-.15in}
\begin{list1}
\item[] Bachelor of Engineering, \href{http://www.extc.fcrit.ac.in/}{\textcolor{blue}{Electronics and Telecommunication Engineering}}, \textit{First Class}
\end{list1}

\vspace*{-0.1in}

{\bf \href{http://www.aejcmumbai.ac.in/}{\textcolor{blue}{Atomic Energy Junior College}}}, Anushaktinagar, Mumbai \hfill {\it 2008 -- 2010} \\
\vspace*{-.15in}
\begin{list1}
\item[] Intermediate Degree, Science with Vocational Training in Electrical Maintenance, \textit{88\%}
\end{list1}

\vspace*{-0.1in}

{\bf Atomic Energy Central School}, Anushaktinagar, Mumbai \hfill {\it 1998 -- 2008} \\
\vspace*{-.15in}
\begin{list1}
\item[] Matriculation, \textit{93.8\%}, \textit{All India Topper in Mathematics}
\end{list1}

\section{\sc Research Experience \\ and Projects} 

{\bf  \href{http://redxmumbai.com/}{\textcolor{blue}{REDX Innovation Lab}}}, under {\bf \href{http://mitemergingworlds.com/}{\textcolor{blue}{Emerging Worlds Special Interest Group}}, MIT Media Lab} \\
{\em Innovation Engineer}, {\em focusing on healthcare technologies for rural areas} \\
\vspace*{-.05in}
\begin{list1}
\item[]\textbf{Project: \href{http://dermato.io/}{\textcolor{blue} {Dermato}}} \hfill $\|$ \hfill {\em Guide: \href{http://www.mit.edu/~ajdas}{\textcolor{blue} {Dr. Anshuman J. Das}}} \hfill $\|$ \hfill {October 2015 -- Present} \\
\textit{A project aiming to solve important problems faced by dermatologists in diagnosis in rural areas}
\vspace{4pt}

\noindent \textbf{The first part} is a study aimed at reducing the subjectivity in visual examination of skin diseases, especially in depigmented conditions. Here, it is difficult to diagnose similar looking diseases without an intravenous procedure like biopsy. We have developed a non-intravenous method to identify the condition which is much faster, cheaper and less painful than existing methods. The device is a fluorescence spectroscopy test that can be done with the help of a mobile phone. I have been involved in collaborating with doctors to design and implement this study at the K. J. Somaiya Hospital, Mumbai. I have been actively helping doctors to collect data using our device, from around 200 skin patients over a period of 5 months. I have been working on analyzing this data and improving the design and user interface based on feedback from doctors. We plan to publish the results of this study at a prominent healthcare conference.
\vspace{4pt}

\noindent \textbf{The second part} of the project is a portable mobile clip-on that can be used by dermatologists in visual examination of the skin. This affordable app-based device provides magnified, high quality and polarized images of skin lesions and allows one to annotate, store and access them along with patient data. I was involved in giving inputs for its design and usability. I was involved in a pilot study that included doctors from K. J. Somaiya hospital to test the device on their patients and give us critical feedback of its design and app interface. Its success led us to mass-manufacture the clip-on, to be distributed to about 500 dermatologists in and around Mumbai. \\
\vspace*{-.05in}

\item[]\textbf{Project: \href{http://dermato.io/}{\textcolor{blue} {StethoCG}}} \hfill $\|$ \hfill {\em Guide: \href{web.media.mit.edu/~guysatat/}{\textcolor{blue} {Guy Satat}}} \hfill $\|$ \hfill {October 2015 -- Present} \\
\textit{A project aiming to provide better detection of cardiovascular disorders in rural areas}
\vspace{4pt}

\noindent Cardiovascular diseases, among all reasons, cause the maximum number ($\sim$ 1.4 million) of deaths per year in India. Rural areas are especially affected due to a lack of doctors specializing in diagnosis of these disorders. We aimed to solve this problem with a screening device that helps in detecting heart murmurs in patients leading to possible further diagnosis and treatment. The device combines a digital stethoscope and an ECG to identify irregularities in heartbeat. I had developed the hardware prototype which led to better data collection. The user interface of the device designed to be simple for use by a health worker and its form factor was reduced to make it portable. This was used to collect data samples from various patients and the preliminary data was analyzed to detect similarities in their pattern.\\ 
This project was \textit{presented to Ratan Tata} and it got positive feedback.
\end{list1}

{\bf Srujana -- Center for Innovation, \href{http://www.lvpei.org/}{\textcolor{blue}{L. V. Prasad Eye Institute}}} \\
{\em Guide: \href{http://www.lvpei.org/our-team/our-team-ashutosh.php}{\textcolor{blue}{Ashutosh Richhariya}}, Ophthalmic Biophysics, LVPEI} \hfill {\it Winter 2014} \\
\vspace*{-.13in}
\begin{list1}
\item[]\textbf{Super--Resolution with Fourier Ptychographic Microscopy} \\
The maximum number of independent samples that an imaging setup can extract from the imaged scene is dictated by the limits set by Fourier optics: if the imaged scene is wide--field, there are limits on how much one can zoom in. However, wide--field, high--resolution images are generally desirable in microscopy, pathology and eye imaging. Traditional methods to achieve this, ptychography, for instance, involve mechanical adjustment and require precise control and actuation. Fourier Ptychographic Microscopy was introduced in 2013 as a computational tool to work around this. The idea is to shift, in the Fourier domain, high--frequency information to low frequencies (where the system's optics do not filter it), sense it, and shift it back while reconstructing. We worked on understanding and implementing this method for microscopy slides, and analyzed possible extensions to imaging reflective surfaces like the eye.
Code \href{https://github.com/alankarkotwal/lvp-imaging/tree/parallel}{\textcolor{blue} {here}}.
\end{list1}

\vspace{-.1cm}

\section{\sc Research Projects}

{\bf A Bayesian Framework for Laparoscopic Image Dehazing and Denoising} \\
{\em Guide: \href{https://www.cse.iitb.ac.in/~suyash}{\textcolor{blue}{Prof. Suyash Awate}}, CSE, IITB} \hfill {\it January 2015 -- Present} \\
\vspace*{-.13in}
\begin{list1}
\item[]
Laparoscopic images in minimally invasive surgery get corrupted by surgical smoke and noise. This degrades the quality of the surgery and the results of further processing for, say, segmentation and tracking. Algorithms for desmoking and denoising laparoscopic images seem to be missing in the medical vision literature. We formulated the problem of joint desmoking and denoising of laparoscopic images as a Bayesian inference problem. This formulation relies on a novel probabilistic graphical model of images, which includes a Markov Random Field (MRF) formulation for color-contrast and another MRF for smoothness on the uncorrupted color image as well as the transmission-map image that indicates color attenuation due to smoke. We resort to a mode approximation (using gradient descent) to the Bayesian posterior to estimate the underlying scene. The results on simulated and real-world laparoscopic images, with clinical expert evaluation, shows the advantages of our method over the state of the art. We now extend this framework to include speckle removal, with better optimization techniques like Variational Bayes--based Expectation Maximization. Code \href{https://github.com/alankarkotwal/lap-dehazing}{\textcolor{blue} {here}}.
\end{list1}

{\bf Optimizing Sensing Matrices for Compressed Sampling Recovery} \hfill \textit{Master's Thesis} \\
{\em Guide: \href{https://www.cse.iitb.ac.in/~ajitvr}{\textcolor{blue}{Prof. Ajit Rajwade}}, CSE \& \href{https://www.ee.iitb.ac.in/wiki/faculty/rajbabu}{\textcolor{blue}{Prof. V. Rajbabu}}, EE, IITB} \hfill {\it December 2015 -- Present} \\
\vspace*{-.13in}
\begin{list1}
\item[]
Recent efforts to apply the principles of compressed sensing to video data involve combining frames into coded snapshots while sensing and separating them with an over--complete dictionary. This works well, but needs a dictionary at the same frame--rate and time--smoothness as the video. We try relaxing this constraint using a source--separation approach to this problem where precise error bounds on recovery have been derived. We designed practically--realizable positive patch--wise sensing matrices that have low mutual coherence in all circular permutations, which facilitates accurate overlapping patchwise recovery. We currently aim to design general sensing matrices with quantities that yield tighter error bounds than the coherence. We thus explore bounds (like the $l_1$--CSMV and $l_\infty$--based bounds) which `soften' the sparsity criterion for $l_1$ recovery, resulting in quantities verifiable in polynomial time, as opposed to the RIC bound which is combinatorial and therefore difficult to optimize. Code \href{https://bitbucket.org/alankarkotwal/coded-sourcesep}{\textcolor{blue} {here}}.
\end{list1}

{\bf The IITB Mars Rover Project}
%\\
%{\em Guide: \href{http://www.aero.iitb.ac.in/~pjguru/}{\textcolor{blue}{Prof. PJ Guruprasad}}, Aerospace, IITB}
\hfill {\it May 2013 -- Present} \\
\vspace*{-.13in}
\begin{list1}
\item[]
The IITB Mars Rover project is a student initiative at IIT Bombay to build a prototype Mars rover capable of extra-terrestrial robotics and to participate in the \href{http://urc.marssociety.org/}{\textcolor{blue} {University Rover Challenge}} at the Mars Society's \href{http://mdrs.marssociety.org/}{\textcolor{blue} {Mars Desert Research Station}}, Utah. The mechanical subsystem designed and developed a rover with a rocker-bogie suspension and novel air-filled beach tires. The electrical and software team designed power, logic and communication circuits for on-board control. Currently, localization and autonomous navigation are being developed. The role of machine vision for automating rover operations is being explored. One of the design goals for the future is to develop the capability to help astronauts on space missions. We participated in a simulated Martian expedition and tested Rover capabilities in the harsh conditions of the Australian outback, at the \href{http://marssociety.org.au/article/arkaroola-mars-robot-challenge-spaceward-bound-expedition}{\textcolor{blue} {Arkaroola Mars Robot Challenge}}. We participated in a series of exercises in Mars Operations Research, involving test extra--vehicular activities in space--suits (a sample collection task and a rover guidance task) conducted by \href{https://saberastro.com/}{\textcolor{blue} {Saber Astronautics}}. Details and pictures \href{http://alankarkotwal.github.io/#projects}{\textcolor{blue} {here}}.
\end{list1}

\section{\sc Course \\Projects}
{\bf Improved Methods for Compressed Sensing Recovery} \hfill {\it CS709: Convex Optimization} \\
{\em Guide: \href{https://www.cse.iitb.ac.in/~ganesh/}{\textcolor{blue}{Prof. Ganesh Ramakrishnan}}, CSE, IITB \hfill Autumn 2015-16} \\
\vspace*{-.15in}
\begin{list1}
\item[] Using convex approximations to the compressed sensing recovery problem, we reconstructed near-exact versions of images at extremely low compressions, with proofs of correctness. Code \href{https://github.com/alankarkotwal/cs-rank-minimization}{\textcolor{blue} {here}}.
\end{list1}

\vspace*{-0.1in}

{\bf Hidden Markov Model Part-of-Speech Tagging} \hfill \textit{EE638: Estimation and Identification} \\
{\em Guide: \href{https://www.ee.iitb.ac.in/course/~ee638/Navin}{\textcolor{blue}{Prof. Navin Khaneja}}, EE, IITB \hfill Autumn 2015-16} \\
\vspace*{-.15in}
\begin{list1}
\item[] We implemented part-of-speech tagging with support for unknown words. An error rate of around 5\% and capabilities of the system to discern context were observed. Code \href{https://github.com/alankarkotwal/pos-tagging}{\textcolor{blue} {here}}.
\end{list1}

\vspace*{-0.1in}

{\bf Laparoscopic Image Dehazing with Dark Channel Prior} \hfill \textit{CS736: Medical Image Processing} \\
{\em Guide: \href{https://www.cse.iitb.ac.in/~suyash}{\textcolor{blue}{Prof. Suyash Awate}}, CSE, IITB \hfill Spring 2014-15} \\
\vspace*{-.15in}
\begin{list1}
\item[] We applied the Dark Channel Prior method for landscape image dehazing to surgical smoke--affected laparoscopic images, accelerated it in time and got good results. Code \href{https://github.com/riddhishb/Laproscopic-Image-Dehazing}{\textcolor{blue} {here}}.
\end{list1}

\vspace*{-0.1in}

{\bf Stereo Odometry via Point Cloud Registration} \hfill \textit{CS763: Computer Vision} \\
{\em Guide: \href{https://www.cse.iitb.ac.in/~ajitvr}{\textcolor{blue}{Prof. Ajit Rajwade}}, CSE, IITB \hfill Spring 2014-15} \\
\vspace*{-.15in}
\begin{list1}
\item[] Maximizing kernel density correlation with gradient-ascent and coherent point drift, we registered pointclouds and observed good convergence behavior for small transformations. Code \href{https://github.com/alankarkotwal/stereo-vo}{\textcolor{blue} {here}}.
\end{list1}

\vspace*{-0.1in}

{\bf Gravitational Lens Separation with PCA} \hfill \textit{CS663: Digital Image Processing} \\
{\em Guide: \href{https://www.cse.iitb.ac.in/~suyash}{\textcolor{blue}{Prof. Suyash Awate}} \& \href{https://www.cse.iitb.ac.in/~ajitvr}{\textcolor{blue}{Prof. Ajit Rajwade}}, CSE, IITB \hfill Autumn 2014-15} \\
\vspace*{-.15in}
\begin{list1}
\item[] Gravitationally lensed images of galaxies have rare arc-like artifacts that can be used to calculate the mass of the lens. We used Anscombe denoising followed by PCA to build a basis for galaxy images and used the top few eigengalaxies to subtract sources and detect arcs. Code \href{https://github.com/alankarkotwal/pca-lens-finder}{\textcolor{blue} {here}}.
\end{list1}

\vspace*{-0.1in}

{\bf Processor Design} \hfill \textit{EE309: Microprocessors} \\
{\em Guide: \href{https://www.ee.iitb.ac.in/~viren/}{\textcolor{blue}{Prof. Virendra Singh}}, EE, IITB \hfill Autumn 2014-15} \\
\vspace*{-.15in}
\begin{list1}
\item[] We designed, simulated and implemented a \href{https://github.com/alankarkotwal/lc-3b-processor}{\textcolor{blue} {multi-cycle RISC processor}} following the LC-3b ISA. Also, we designed and simulated a \href{https://github.com/alankarkotwal/lca-processor}{\textcolor{blue} {pipelined RISC processor}} using the Little Computer ISA.
\end{list1}

\section{\sc Astrophysics \\Projects}
{\bf Detecting Short $\gamma$-ray Bursts in Astrosat CZTI Data} \hfill \textit{PH426: Astrophysics} \\
{\em Guide: \href{https://sites.google.com/site/vikramrentalahome/}{\textcolor{blue}{Prof. Vikram Rentala}}, \textit{PH, IITB} and \href{http://web.tifr.res.in/~arrao/}{\textcolor{blue}{Prof. A. R. Rao}}, \href{http://www.tifr.res.in/}{\textcolor{blue} {TIFR, Mumbai}} \hfill Spring 2015-16} \\
\vspace*{-.15in}
\begin{list1}
\item[] We did a literature survey on $\gamma$-ray bursts, including open problems in the field. We tackle detecting short $\gamma$-ray bursts from data acquired by the CZTI X-Ray Imager on-board Astrosat.
\end{list1}

\vspace*{-0.1in}

{\bf Variability Analysis for Globular Cluster NGC2419} \hfill \textit{\href{http://nius.hbcse.tifr.res.in/}{\textcolor{blue} {NIUS, Astronomy}}} \\
{\em Guide: \href{http://http://manuu.ac.in/deptphysc_faclty.php/}{\textcolor{blue}{Prof. Priya Hasan}}, \href{http://manuu.ac.in/}{\textcolor{blue} {MANUU, Hyderabad}} \hfill December 2015} \\
\vspace*{-.15in}
\begin{list1}
\item[] We analyzed raw data for the globular cluster NGC2419 taken at the \href{http://www.iiap.res.in/iao/cycle.html}{\textcolor{blue} {HCT}}, post-processed it to correct for detector bias and flat-fielding, inverted the effect of atmospheric mass and extracted the variation of magnitudes of stars in the cluster on the scale of a day. Code \href{https://github.com/alankarkotwal/ngc2419-variables}{\textcolor{blue} {here}}.
\end{list1}

\vspace*{-0.1in}

{\bf An X-ray Study of Black Hole Candidate X Norma X-1} \hfill \textit{\href{http://nius.hbcse.tifr.res.in/}{\textcolor{blue} {NIUS, Astronomy}}} \\
{\em Guide: \href{http://cbs.ac.in/people/visiting-scientists/manojendu-choudhury}{\textcolor{blue}{Prof. Manojendu Choudhury}}, \href{http://cbs.ac.in/}{\textcolor{blue} {Center for Basic Sciences}} \hfill December 2013} \\
\vspace*{-.15in}
\begin{list1}
\item[] We analyzed data from the RXTE for a low-mass X-Ray Binary. Fitting 3-30 keV spectra with a model accounting for blackbody \& non-thermal radiation, and interstellar extinction, we obtained values of system parameters like internal radius and temperature. Report \href{https://alankarkotwal.github.io/4U_1630-47_Report.pdf}{\textcolor{blue} {here}}.
\end{list1}

\vspace*{-0.1in}

{\bf Estimation of Photometric Redshifts with Machine Learning} \hfill \textit{\href{http://nius.hbcse.tifr.res.in/}{\textcolor{blue} {NIUS, Astronomy}}} \\
{\em Guide: \href{http://www.iucaa.ernet.in/~nspp/}{\textcolor{blue}{Prof. Ninan Sajeeth Philip}}, \href{http://www.iucaa.ernet.in/}{\textcolor{blue} {IUCAA}}, Pune \hfill December 2012} \\
\vspace*{-.15in}
\begin{list1}
\item[] Here, we trained a neural network for photometric redshifts, given data for sources whose spectra and redshifts have been measured. We predicted spectra for these objects viewed at various other values of redshifts. Using this expanded dataset, we achieved good predictions for test data.
\end{list1}

\section{\sc Achievements \\ and Awards}
\begin{list2}
\item Represented India at the \href{http://www.ioaa2012.ufrj.br/}{\textcolor{blue} {$6^\text{th}$ International Olympiad on Astronomy and Astrophysics}}, Brazil, 2012. Won a Gold Medal with International Rank 4 and a special prize for Best Data Analysis
\item Represented India at the \href{http://www.ieso2011.unimore.it/}{\textcolor{blue} {$5^\text{th}$ International Earth Sciences Olympiad}}, Italy, 2011. Won a Bronze Medal and prizes for best performance in the Hydrosphere section and the team presentation
\item  Secured all-India rank $105$ in the \href{https://en.wikipedia.org/wiki/Indian_Institute_of_Technology_Joint_Entrance_Examination}{\textcolor{blue}{IIT-JEE}} 2012 amongst half a million candidates
\item Awarded \href{http://www.kvpy.iisc.ernet.in/main/index.htm}{\textcolor{blue}{KVPY Scholarship}} $2011$ by Dept. of Science and Technology, Govt. of India
\item Awarded \href{http://www.ncert.nic.in/programmes/talent_exam/index_talent.html}{\textcolor{blue}{NTSE Scholarship}} $2008$ by NCERT, Govt. of India
\end{list2}

\section{\sc Key Talks \\ and Seminars}
{\bf Coded Source Separation for Compressed Video Recovery} \hfill {\em Master's Thesis Talk} \\
{\em \href{http://www.ee.iitb.ac.in/}{\textcolor{blue}{Department of Electrical Engineering}}, \href{http://www.iitb.ac.in/}{\textcolor{blue}{Indian Institute of Technology Bombay}} \hfill May 2016} \\
\vspace*{-.15in}
\begin{list1}
\item[] Here, I presented results from the first stage of my dual degree thesis. Presentation \href{http://alankarkotwal.github.io/sre.pptx}{\textcolor{blue} {here}}.
\end{list1}

\vspace*{-0.1in}

{\bf Template-Based Stereo Odometry} \hfill {\em Invited Talk} \\
{\em \href{http://theairlab.org/}{\textcolor{blue}{The AIR Lab}}, \href{http://www.cmu.edu/}{\textcolor{blue}{Carnegie Mellon University}} \hfill July 2015} \\
\vspace*{-.15in}
\begin{list1}
\item[] I presented results of summer internship at CMU. The talk included a detailed description of the method used, comparisons with ground-truth and stress-tests on the method. Presentation \href{http://alankarkotwal.github.io/intern_presentation.pptx}{\textcolor{blue} {here}}.
\end{list1}

\vspace*{-0.1in}

{\bf \href{http://www.stab-iitb.org/krittika/the-cosmic-ladder-distance}{\textcolor{blue} {The Cosmic Distance Ladder}}} \hfill {\em Invited Talk} \\
{\em \href{http://www.stab-iitb.org/krittika/}{\textcolor{blue} {Krittika -- The Astronomy Club, IIT Bombay}} \hfill September 2014, February 2016, August 2016} \\
\vspace*{-.15in}
\begin{list1}
\item[] This talk climbs the Cosmic Distance Ladder, a sequence of steps, each building on previous steps' results, for calculating distances in the universe. We begin with solar system distances, and end at enormous distances where the only option is using indirect methods. Presentation \href{http://alankarkotwal.github.io/CosmicDistanceLadder.pptx}{\textcolor{blue} {here}}.
\end{list1}

\section{\sc Mentoring Experience}
\textbf{Teaching Assistant for IITB Courses}
\begin{list1}
\item[] CS663: Digital Image Processing \hspace{0.5cm} \href{https://www.cse.iitb.ac.in/~suyash}{\textcolor{blue}{Prof. S. Awate}} \& \href{https://www.cse.iitb.ac.in/~ajitvr}{\textcolor{blue}{Prof. A. Rajwade}} \hfill{\textit{Autumn 2015-16}}
\item[] CS736: Medical Image Processing \hspace{2cm} \href{https://www.cse.iitb.ac.in/~suyash}{\textcolor{blue}{Prof. S. Awate}}\hfill{\textit{Spring 2015-16}}
\item[] EE638: Estimation and Identification \hspace{1.25cm} \href{https://www.ee.iitb.ac.in/course/~ee638/Navin}{\textcolor{blue}{Prof. N. Khaneja}}\hfill{\textit{Autumn 2016-17}}
\end{list1}

\vspace*{-0.1in}

\textbf{Resource Person, Indian Astronomy Olympiad Programme} \hfill \textit{May 2013, May 2014} \\
\vspace*{-.15in}
\begin{list1}
\item[] Selected twice as a resource person for Indian Astronomy Olympiad Camps, to mentor students for their selection to the International Astronomy Olympiads. Involved in mentoring 100-odd high school students in astronomy, and in setting up challenging questions and evaluating them.
\end{list1}

\section{\sc Key \\Coursework} 
\begin{list1}
\item[\strut\hspace{0.5cm}\hypertarget{crselst}{\textbf{Computer Sciences and Engineering}}]
\item[]\textit{Computer Vision, Digital \& Medical Image Processing, Machine Learning, Convex Optimization, Graphics, Networks, Data Structures, Design \& Analysis of Algorithms, Discrete Mathematics}
\item[\strut\hspace{0.5cm}\textbf{Electrical Engineering}]
\vspace{0.05in}
\item[]\textit{Estimation \& Identification, Adaptive \& Digital Signal Processing, Speech Processing, Matrix Computations, Information Theory, Advanced Probability, Communication Networks}
\item[\strut\hspace{0.5cm}\textbf{Physics and Mathematics}]
\vspace{0.05in}
\item[]\textit{Astrophysics, General Theory of Relativity, Electromagnetic Waves, Electricity \& Magnetism, Classical Mechanics, Differential Equations, Linear Algebra, Complex Analysis, Calculus}
\end{list1}

\section{\sc Technical \\Skills} 
\begin{tabular}{@{}p{1.3in}p{4.3in}}
\textbf{Programming} & C/C++, Python, Bash, Matlab, Verilog, SQL, HTML/CSS, PHP, \LaTeX \\  
\vspace*{-0.06in}
\textbf{Software Packages} & 
\vspace*{-0.06in}
ROS/Gazebo, OpenCV, The Point Cloud Library \\ 
\vspace*{-0.06in}
\textbf{Science Software} &
\vspace*{-0.06in}
Python packages: NumPy, SciPy and Matplotlib, GNUPlot, Scikit-learn \\
\vspace*{-0.06in}
\textbf{Hardware} &
\vspace*{-0.06in}
\textit{Microprocessors:} 8051, 8085, AVR and PIC, CPLDs and FPGAs, \textit{Embedded Platforms:} Arduino, Raspberry Pi, NVIDIA Jetson TK1 \\     
\end{tabular}

\section{\sc Other \\Interests}
\lettrine[lines=2]{O}{ther} than my academic interests, I like biking, long walks, trekking, climbing whatever can be climbed, swimming, table tennis, socializing, cooking good food and eating it. I especially enjoy classic rock music \& people who enjoy my interests. And finally, I love Origami, the art of paper folding, and building complex, realistic models with Lego blocks.

\section{\sc References}
\begin{tabular}{@{}p{3in}p{3in}}
\textbf{Guy Satat}, Camera Culture & \textbf{Prof. Ajit Rajwade}, CSE \\ 
MIT Media Lab $|$ \href{mailto:guysatat@mit.edu}{\textcolor{blue}{E--Mail}} $|$ \href{http://web.media.mit.edu/~guysatat/}{\textcolor{blue}{Webpage}} & IITB $|$ \href{mailto:ajitvr@cse.iitb.ac.in}{\textcolor{blue}{E--Mail}} $|$ \href{https://www.cse.iitb.ac.in/~ajitvr}{\textcolor{blue}{Webpage}} \\
\end{tabular}
\vspace{-0.15in}

\begin{tabular}{@{}p{3in}p{3in}}
\textbf{Dr. Sebastian Scherer}, Robotics Institute & \textbf{Ashutosh Richhariya}, Ophthalmic Biophysics \\ 
CMU $|$ \href{mailto:basti@andrew.cmu.edu}{\textcolor{blue}{E--Mail}} $|$ \href{http://www.ri.cmu.edu/person.html?person_id=1397}{\textcolor{blue}{Webpage}} & LVPEI $|$ \href{mailto:ashutosh@lvpei.org}{\textcolor{blue}{E--Mail}} $|$ \href{http://www.lvpei.org/our-team/our-team-ashutosh.php}{\textcolor{blue}{Webpage}} \\
\end{tabular}
\vspace{-0.15in}

\begin{tabular}{@{}p{3in}p{3in}}
\textbf{Prof. Mayank Vahia}, Astrophysics & \textbf{Dr. Aniket Sule}, Astronomy \\
TIFR $|$ \href{mailto:vahia@tifr.res.in}{\textcolor{blue}{E--Mail}} $|$ \href{http://www.tifr.res.in/~vahia/}{\textcolor{blue}{Webpage}} & HBCSE--TIFR $|$ \href{mailto:anikets@hbcse.tifr.res.in}{\textcolor{blue}{E--Mail}} $|$ \href{http://www.hbcse.tifr.res.in/people/academic/aniket-sule}{\textcolor{blue}{Webpage}} \\
\end{tabular}
\vspace{-0.15in}

\begin{tabular}{@{}p{3in}p{3in}}
\textbf{Prof. Rajbabu Velmurugan}, EE & \textbf{Dr. Manojendu Choudhury}, Astrophysics \\
IITB $|$ \href{mailto:rajbabu@ee.iitb.ac.in}{\textcolor{blue}{E--Mail}} $|$ \href{https://www.ee.iitb.ac.in/web/faculty/homepage/rajbabu}{\textcolor{blue}{Webpage}} & UM--DAE CBS $|$ \href{mailto:manojendu@cbs.ac.in}{\textcolor{blue}{E--Mail}} $|$ \href{http://www.cbs.ac.in/people/physics-faculty/manojendu-choudhury}{\textcolor{blue}{Webpage}} \\
\end{tabular}
\vspace{-0.15in}

%\begin{tabular}{@{}p{3in}p{3in}}
%\textbf{Prof. Rajbabu Velmurugan}, EE & \textbf{Dr. Manojendu Choudhury}, Astrophysics \\
%IITB $|$ \href{mailto:rajbabu@ee.iitb.ac.in}{\textcolor{blue}{E--Mail}} $|$ \href{https://www.ee.iitb.ac.in/web/faculty/homepage/rajbabu}{\textcolor{blue}{Webpage}} & UM--DAE CBS $|$ \href{mailto:manojendu@cbs.ac.in}{\textcolor{blue}{E--Mail}} $|$ \href{http://www.cbs.ac.in/people/physics-faculty/manojendu-choudhury}{\textcolor{blue}{Webpage}} \\
%\end{tabular}

\end{resume}
\end{document}