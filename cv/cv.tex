\documentclass[margin,line]{res}

\usepackage{hyperref}
\usepackage{amsmath}
\usepackage{textcomp}
\usepackage{color}
\usepackage{lettrine}
%\usepackage[left=0.75in, right=0.75in, top=0.75in,bottom=0.75in, margin=0.75in]{geometry}
\oddsidemargin = -.5in
\evensidemargin = -.5in
\topmargin = -0.3in
\textheight = 9.75in

\textwidth = 6.0in
\itemsep=0in
\parsep=0in
% if using pdflatex:
%\setlength{\pdfpagewidth}{\paperwidth}
%\setlength{\pdfpageheight}{\paperheight} 

\newenvironment{list1}{
  \begin{list}{\ding{113}}{%
      \setlength{\itemsep}{0in}
      \setlength{\parsep}{0in} \setlength{\parskip}{0in}
      \setlength{\topsep}{0in} \setlength{\partopsep}{0in} 
      \setlength{\leftmargin}{0.17in}}}{\end{list}}
\newenvironment{list2}{
  \begin{list}{$\bullet$}{%
      \setlength{\itemsep}{0in}
      \setlength{\parsep}{0in} \setlength{\parskip}{0in}
      \setlength{\topsep}{0in} \setlength{\partopsep}{0in} 
      \setlength{\leftmargin}{0.2in}}}{\end{list}}

\tolerance=1
\emergencystretch=\maxdimen
\hyphenpenalty=10000
\hbadness=10000

\pagenumbering{gobble}

\begin{document}

%\name{\textbf{\huge{Alankar Kotwal}} \vspace*{.05in} \newline  {\sc Senior Undergraduate}\vspace*{.1in}}
\name{\textbf{\huge{Aparajita Sahoo}} \vspace*{.1in} }

\begin{resume}
\section{\sc Contact Information}
\vspace{.05in}
\begin{tabular}{@{}p{2.9in}p{.5in}p{3in}}
\href{https://redxmumbai.com/}{\textcolor{blue} {REDX Innovation Lab, Mumbai}} & \multicolumn{1}{r}{\it Phone:}  &(+91) 907 621 1711 \\           
\href{http://redx.io/}{\textcolor{blue} {Rethinking Engineering Design eXecution}} &\multicolumn{1}{r}{\it E--Mail:}& \href{mailto:aparajitasahoois@gmail.com}{\textcolor{blue}{aparajitasahoois@gmail.com}} \\ 
Welingkar Building, Lakhamsi Nappu Road & & \href{mailto:aparajita.redx@gmail.com}{\textcolor{blue}{aparajita.redx@gmail.com}} \\ 
Matunga (East), Mumbai, India 400 019 & \multicolumn{1}{r}{\it Webpage:} &\href{http://aparajitasahoo.github.io/}{\textcolor{blue}{aparajitasahoo.github.io}} \\     
\end{tabular}

\section{\sc Research Interests}
\lettrine[lines=2]{I}{} am passionate about creating a social impact by working at the intersection of health technologies, data science, design and social innovation that will impact the lives of the people globally. I am also interested in enhancing  user experiences by combining the power of computing, mobile technologies and effective design. I have also worked on electronics prototyping of low cost healthcare devices that can be deployed in rural areas that don’t have access to medical facilities.  

\section{\sc Education}
{\bf \href{http://mu.ac.in/portal/}{\textcolor{blue}{Univ. of Mumbai}}, \href{http://fcrit.ac.in/}{\textcolor{blue}{Fr. Conceicao Rodrigues Inst. of Technology}}}, Mumbai \hfill {\it 2011 -- 2015} \\
\vspace*{-.15in}
\begin{list1}
\item[] Bachelor of Engineering, \href{http://www.extc.fcrit.ac.in/}{\textcolor{blue}{Electronics and Telecommunication Engineering}}, \textit{First Class}
\end{list1}

\vspace*{-0.1in}

{\bf \href{http://www.aejcmumbai.ac.in/}{\textcolor{blue}{Atomic Energy Junior College}}}, Anushaktinagar, Mumbai \hfill {\it 2008 -- 2010} \\
\vspace*{-.15in}
\begin{list1}
\item[] Intermediate Degree, Science with Vocational Training in Electrical Maintenance, \textit{88\%}
\end{list1}

\vspace*{-0.1in}

{\bf Atomic Energy Central School}, Anushaktinagar, Mumbai \hfill {\it 1998 -- 2008} \\
\vspace*{-.15in}
\begin{list1}
\item[] Matriculation, \textit{93.8\%}, \textit{All India Topper in Mathematics}
\end{list1}

\section{\sc Research Experience \\ and Projects} 

{\bf  \href{http://redxmumbai.com/}{\textcolor{blue}{REDX Innovation Lab}}}, under {\bf \href{http://mitemergingworlds.com/}{\textcolor{blue}{Emerging Worlds Special Interest Group}}, \href{https://www.media.mit.edu}{\textcolor{blue} {MIT Media Lab}}} \\
{\em Innovation Engineer}, {\em focusing on healthcare technologies for rural areas} \\
\vspace*{-.05in}
\begin{list1}
\item[]\textbf{Project: \href{http://dermato.io/}{\textcolor{blue} {Dermato}}} \hfill $\|$ \hfill {\em Guide: \href{http://www.mit.edu/~ajdas}{\textcolor{blue} {Dr. Anshuman J. Das}}} \hfill $\|$ \hfill {Oct '15 -- Present} \\
\textit{A project aiming to solve important problems faced by dermatologists in diagnosis in rural areas}
\vspace{4pt}

\noindent \textbf{The first part} is a study aimed at reducing the subjectivity in visual examination of skin diseases, especially in depigmented conditions. Here, it is difficult to diagnose similar looking diseases without an intravenous procedure like biopsy. We have developed a non-intravenous method to identify the condition which is much faster, cheaper and less painful than existing methods. The device is a fluorescence spectroscopy test that can be done with the help of a mobile phone. I have been involved in collaborating with doctors to design and implement this study at the K. J. Somaiya Hospital, Mumbai. I have been actively helping doctors to collect data using our device, from around 200 skin patients over a period of 5 months. I have been working on analyzing this data and improving the design and user interface based on feedback from doctors. We plan to publish the results of this study at a prominent healthcare conference.
\vspace{4pt}

\noindent \textbf{The second part} of the project is a portable mobile clip-on that can be used by dermatologists in visual examination of the skin. This affordable app-based device provides magnified, high quality and polarized images of skin lesions and allows one to annotate, store and access them along with patient data. I was involved in giving inputs for its design and usability. I was involved in a pilot study that included doctors from K. J. Somaiya hospital to test the device on their patients and give us critical feedback of its design and app interface. Its success led us to mass-manufacture the clip-on, to be distributed to about 500 dermatologists in and around Mumbai. \\
\vspace*{-.05in}

\item[]\textbf{Project: StethoCG} \hfill $\|$ \hfill {\em Guide: \href{http://web.media.mit.edu/~guysatat/}{\textcolor{blue} {Guy Satat}}} \hfill $\|$ \hfill {Jun '13 -- Oct '13} \\
\textit{A project aiming to provide better detection of cardiovascular disorders in rural areas}
\vspace{4pt}

\noindent Cardiovascular diseases, among all reasons, cause the maximum number ($\sim$ 1.4 million) of deaths per year in India. Rural areas are especially affected due to a lack of doctors specializing in diagnosis of these disorders. We aimed to solve this problem with a screening device that helps in detecting heart murmurs in patients leading to possible further diagnosis and treatment. The device combines a digital stethoscope and an ECG to identify irregularities in heartbeat. I had developed the hardware prototype which led to better data collection. The user interface of the device designed to be simple for use by a health worker and its form factor was reduced to make it portable. This was used to collect data samples from various patients and the preliminary data was analyzed to detect similarities in their pattern.\\ 
This project was \textit{presented to Ratan Tata} and it got positive feedback.

\item[]\textbf{Project: Pediatric Perimeter} \hfill $\|$ \hfill {\em Guide: \href{http://www.lvpei.org/our-team/our-team-Premnandhini.php}{\textcolor{blue} {Dr. Premnandhini Satgunam}}} \hfill $\|$ \hfill {Jun '13 -- Oct '13} \\
\textit{A project aiming test for visual field in infants with cerebral palsy} 
\vspace{4pt}

\noindent Peripheral vision impairment is an indicator of potential abnormalities in the human body. It is a challenge to measure the visual field of an infant suffering from physical disabilities like cerebral palsy where the body's movement is restricted and the patient cannot be tested in the standard way. Pediatric Perimeter is a novel device to measure and quantify visual fields and reaction times to light stimulus in infants. This device assists in the early detection of neonatal eye diseases and early signs of vision-threatening conditions. 
The prototype we made recorded the response and the movement of pupil. I was a part of the initial team that had started the project and was involved in hardware prototyping and design of the enclosure. We had tested it on few babies and got valuable feedback. 

\noindent The first version was presented to Dr. A. P. J. Abdul Kalam and received encouraging reviews. 
\end{list1}

\section{\sc Internships}

{\bf LVP -- MITRA Fellow, Srujana -- Centre of Innovation} \hfill  \\
{\em Guides: \href{https://www.media.mit.edu/~sssinha}{\textcolor{blue}{Shantanu Sinha}} , \href{https://web.media.mit.edu/~tswedish/}{\textcolor{blue}{Tristan Swedish}}, \href{https://www.linkedin.com/in/derbedhruv}{\textcolor{blue}{Dhruv Joshi}}} \hfill {Dec '14} \\
\vspace*{-.13in}
\begin{list1}
\item[]
I worked on the hardware prototyping of a modular eye diagnostic device that could be used for tracking the movements of pupil. Its design was inspired by the Google Cardboard. 
\end{list1}

{\bf Health Tech Intern, MIT Rethinking Diagnostics, IIT Bombay} \\
{\em Guide: \href{http://web.media.mit.edu/~guysatat/}{\textcolor{blue}{Guy Satat}}} \hfill {Jun '14} \\
\vspace*{-.13in}
\begin{list1}
\item[]
I worked on developing the initial prototype of StethoCG and tested the initial circuits. The design and form factor was made small so that it could be carried easily by a physician. I took up this project again when I joined the REDX Innovation Lab, Mumbai for further development in its design and data collecting capabilities.
\end{list1}

{\bf Product Development Intern, Qyuki, Mumbai} \\
{\em Guides: \href{http://www.kshitijmarwah.com}{\textcolor{blue}{Kshitij Marwah}} and \href{https://www.linkedin.com/in/ananddhople}{\textcolor{blue}{Anand Dhople}}} \hfill {Sep '13 -- Nov '13} \\
\vspace*{-.13in}
\begin{list1}
\item[]
Qyuki, a startup by Oscar winner A. R. Rahman, film maker Shekhar Kapur and Samir Bangara, is a network that brings together artists to collaborate and create content. I worked on its product detailing and wireframing of the website to help in connecting the creators. I was also involved in designing and testing the website interface. This got me interested in designing user interfaces. 
\end{list1}

\section{\sc Leadership and Initiatives}

{\bf Founding Members of REDX Innovation Lab} \hfill {May '15} \\
\vspace*{-.13in}
\begin{list1}
\item[]
I was among the first members hired for the lab by the Camera Culture group at the MIT Media Lab. This experience of setting up an Innovation Lab from scratch, being involved in critical decisions that would impact the way our lab functioned, managing technical and administrative issues, building an active community of innovators, doctors and corporates around us, collaborating with hospitals to implement our research studies, organizing monthly meetups, building a network of manufacturers and suppliers and designing our products for deployment has been an incredible journey for me. I have learnt a lot about the health innovation ecosystem by interacting regularly with start-ups, clinicians and patients.
\end{list1}

{\bf Starting Maker's Asylum, a Community Makerspace in Mumbai} \hfill {Nov  '13} \\
\vspace*{-.13in}
\begin{list1}
\item[]
I was a part of a group of DIY enthusiasts in Mumbai and our passion for tinkering led us to start our own makerspace called Maker's Asylum which kept growing to become one the best community makerspaces in India.
\end{list1}

{\bf Director Of Operations: IEEE Students Chapter -- FCRIT} \hfill {Jun '13 -- Apr '14} \\
\vspace*{-.13in}
\begin{list1}
\item[]
I held this position in my pre final year where I was actively involved in getting new members to join our IEEE Students Chapter. I was involved in interacting with junior students. I also helped in organizing IEEE competitions and technical festivals.
\end{list1}

{\bf Member of the IETE-ETSA FCRIT Student Council} \hfill {Jun '12 -- Apr '13} \\
\vspace*{-.13in}
\begin{list1}
\item[]
It is the intra-departmental student council that manages the departmental fests and competitions. I was a co-opted member of this council and actively took part in organizing and managing various events during my tenure.
\end{list1}

\section{\sc Other \\Projects}
{\bf Detecting Short $\gamma$-ray Bursts in Astrosat CZTI Data} \hfill \textit{PH426: Astrophysics} \\
{\em Guide: \href{https://sites.google.com/site/vikramrentalahome/}{\textcolor{blue}{Prof. Vikram Rentala}}, \textit{PH, IITB} and \href{http://web.tifr.res.in/~arrao/}{\textcolor{blue}{Prof. A. R. Rao}}, \href{http://www.tifr.res.in/}{\textcolor{blue} {TIFR, Mumbai}} \hfill Spring 2015-16} \\
\vspace*{-.15in}
\begin{list1}
\item[] We did a literature survey on $\gamma$-ray bursts, including open problems in the field. We tackle detecting short $\gamma$-ray bursts from data acquired by the CZTI X-Ray Imager on-board Astrosat.
\end{list1}

\vspace*{-0.1in}

{\bf Variability Analysis for Globular Cluster NGC2419} \hfill \textit{\href{http://nius.hbcse.tifr.res.in/}{\textcolor{blue} {NIUS, Astronomy}}} \\
{\em Guide: \href{http://http://manuu.ac.in/deptphysc_faclty.php/}{\textcolor{blue}{Prof. Priya Hasan}}, \href{http://manuu.ac.in/}{\textcolor{blue} {MANUU, Hyderabad}} \hfill December 2015} \\
\vspace*{-.15in}
\begin{list1}
\item[] We analyzed raw data for the globular cluster NGC2419 taken at the \href{http://www.iiap.res.in/iao/cycle.html}{\textcolor{blue} {HCT}}, post-processed it to correct for detector bias and flat-fielding, inverted the effect of atmospheric mass and extracted the variation of magnitudes of stars in the cluster on the scale of a day. Code \href{https://github.com/alankarkotwal/ngc2419-variables}{\textcolor{blue} {here}}.
\end{list1}

\vspace*{-0.1in}

{\bf An X-ray Study of Black Hole Candidate X Norma X-1} \hfill \textit{\href{http://nius.hbcse.tifr.res.in/}{\textcolor{blue} {NIUS, Astronomy}}} \\
{\em Guide: \href{http://cbs.ac.in/people/visiting-scientists/manojendu-choudhury}{\textcolor{blue}{Prof. Manojendu Choudhury}}, \href{http://cbs.ac.in/}{\textcolor{blue} {Center for Basic Sciences}} \hfill December 2013} \\
\vspace*{-.15in}
\begin{list1}
\item[] We analyzed data from the RXTE for a low-mass X-Ray Binary. Fitting 3-30 keV spectra with a model accounting for blackbody \& non-thermal radiation, and interstellar extinction, we obtained values of system parameters like internal radius and temperature. Report \href{https://alankarkotwal.github.io/4U_1630-47_Report.pdf}{\textcolor{blue} {here}}.
\end{list1}

\vspace*{-0.1in}

{\bf Estimation of Photometric Redshifts with Machine Learning} \hfill \textit{\href{http://nius.hbcse.tifr.res.in/}{\textcolor{blue} {NIUS, Astronomy}}} \\
{\em Guide: \href{http://www.iucaa.ernet.in/~nspp/}{\textcolor{blue}{Prof. Ninan Sajeeth Philip}}, \href{http://www.iucaa.ernet.in/}{\textcolor{blue} {IUCAA}}, Pune \hfill December 2012} \\
\vspace*{-.15in}
\begin{list1}
\item[] Here, we trained a neural network for photometric redshifts, given data for sources whose spectra and redshifts have been measured. We predicted spectra for these objects viewed at various other values of redshifts. Using this expanded dataset, we achieved good predictions for test data.
\end{list1}

\section{\sc Key Talks \\ and Seminars}
{\bf Coded Source Separation for Compressed Video Recovery} \hfill {\em Master's Thesis Talk} \\
{\em \href{http://www.ee.iitb.ac.in/}{\textcolor{blue}{Department of Electrical Engineering}}, \href{http://www.iitb.ac.in/}{\textcolor{blue}{Indian Institute of Technology Bombay}} \hfill May 2016} \\
\vspace*{-.15in}
\begin{list1}
\item[] Here, I presented results from the first stage of my dual degree thesis. Presentation \href{http://alankarkotwal.github.io/sre.pptx}{\textcolor{blue} {here}}.
\end{list1}

\vspace*{-0.1in}

{\bf Template-Based Stereo Odometry} \hfill {\em Invited Talk} \\
{\em \href{http://theairlab.org/}{\textcolor{blue}{The AIR Lab}}, \href{http://www.cmu.edu/}{\textcolor{blue}{Carnegie Mellon University}} \hfill July 2015} \\
\vspace*{-.15in}
\begin{list1}
\item[] I presented results of summer internship at CMU. The talk included a detailed description of the method used, comparisons with ground-truth and stress-tests on the method. Presentation \href{http://alankarkotwal.github.io/intern_presentation.pptx}{\textcolor{blue} {here}}.
\end{list1}

\vspace*{-0.1in}

{\bf \href{http://www.stab-iitb.org/krittika/the-cosmic-ladder-distance}{\textcolor{blue} {The Cosmic Distance Ladder}}} \hfill {\em Invited Talk} \\
{\em \href{http://www.stab-iitb.org/krittika/}{\textcolor{blue} {Krittika -- The Astronomy Club, IIT Bombay}} \hfill September 2014, February 2016, August 2016} \\
\vspace*{-.15in}
\begin{list1}
\item[] This talk climbs the Cosmic Distance Ladder, a sequence of steps, each building on previous steps' results, for calculating distances in the universe. We begin with solar system distances, and end at enormous distances where the only option is using indirect methods. Presentation \href{http://alankarkotwal.github.io/CosmicDistanceLadder.pptx}{\textcolor{blue} {here}}.
\end{list1}

\vspace*{-0.1in}

\section{\sc Key \\Coursework} 
\begin{tabular}{ccccc}
Computer Programming & $|$ & Signal Processing & $|$ & Computer Networks \\
Machine Learning (Coursera) & $|$ & Digital Image Processing & $|$ & Wireless Networks \\
Java & $|$ & Interaction Design (Coursera) & $|$ & Mobile Communication
\end{tabular}

\section{\sc Technical \\Skills} 
\begin{tabular}{@{}p{1.3in}p{4.3in}}
\textbf{Software} & Matlab, R, Python, C, C++, Java, Solidworks, \LaTeX \\  
\vspace*{-0.06in}
\textbf{Hardware} & 
\vspace*{-0.06in}
Arduino, Teensy, Olimex, IoT development boards like Linkit \\ 
\end{tabular}

\section{\sc Other \\Interests}
\lettrine[lines=2]{I}{} like reading books, listening to music and watching movies. I like doodling in my spare time. I am a foodie and like travelling. 

\section{\sc References}
\begin{tabular}{@{}p{3in}p{3in}}
\textbf{Guy Satat}, Camera Culture & \textbf{Shantanu Sinha}, Camera Culture \\ 
MIT Media Lab $|$ \href{mailto:guysatat@mit.edu}{\textcolor{blue}{E--Mail}} $|$ \href{http://web.media.mit.edu/~guysatat/}{\textcolor{blue}{Webpage}} & Meta Vision $|$ \href{mailto:s.sinha@metavision.com}{\textcolor{blue}{E--Mail}} $|$ \href{https://www.media.mit.edu/~sssinha}{\textcolor{blue}{Webpage}} \\
\end{tabular}
\vspace{-0.15in}

%\begin{tabular}{@{}p{3in}p{3in}}
%\textbf{Dr. Sebastian Scherer}, Robotics Institute & \textbf{Ashutosh Richhariya}, Ophthalmic Biophysics \\ 
%CMU $|$ \href{mailto:basti@andrew.cmu.edu}{\textcolor{blue}{E--Mail}} $|$ \href{http://www.ri.cmu.edu/person.html?person_id=1397}{\textcolor{blue}{Webpage}} & LVPEI $|$ \href{mailto:ashutosh@lvpei.org}{\textcolor{blue}{E--Mail}} $|$ \href{http://www.lvpei.org/our-team/our-team-ashutosh.php}{\textcolor{blue}{Webpage}} \\
%\end{tabular}
%\vspace{-0.15in}
%
%\begin{tabular}{@{}p{3in}p{3in}}
%\textbf{Prof. Mayank Vahia}, Astrophysics & \textbf{Dr. Aniket Sule}, Astronomy \\
%TIFR $|$ \href{mailto:vahia@tifr.res.in}{\textcolor{blue}{E--Mail}} $|$ \href{http://www.tifr.res.in/~vahia/}{\textcolor{blue}{Webpage}} & HBCSE--TIFR $|$ \href{mailto:anikets@hbcse.tifr.res.in}{\textcolor{blue}{E--Mail}} $|$ \href{http://www.hbcse.tifr.res.in/people/academic/aniket-sule}{\textcolor{blue}{Webpage}} \\
%\end{tabular}
%\vspace{-0.15in}
%
%\begin{tabular}{@{}p{3in}p{3in}}
%\textbf{Prof. Rajbabu Velmurugan}, EE & \textbf{Dr. Manojendu Choudhury}, Astrophysics \\
%IITB $|$ \href{mailto:rajbabu@ee.iitb.ac.in}{\textcolor{blue}{E--Mail}} $|$ \href{https://www.ee.iitb.ac.in/web/faculty/homepage/rajbabu}{\textcolor{blue}{Webpage}} & UM--DAE CBS $|$ \href{mailto:manojendu@cbs.ac.in}{\textcolor{blue}{E--Mail}} $|$ \href{http://www.cbs.ac.in/people/physics-faculty/manojendu-choudhury}{\textcolor{blue}{Webpage}} \\
%\end{tabular}
%\vspace{-0.15in}

%\begin{tabular}{@{}p{3in}p{3in}}
%\textbf{Prof. Rajbabu Velmurugan}, EE & \textbf{Dr. Manojendu Choudhury}, Astrophysics \\
%IITB $|$ \href{mailto:rajbabu@ee.iitb.ac.in}{\textcolor{blue}{E--Mail}} $|$ \href{https://www.ee.iitb.ac.in/web/faculty/homepage/rajbabu}{\textcolor{blue}{Webpage}} & UM--DAE CBS $|$ \href{mailto:manojendu@cbs.ac.in}{\textcolor{blue}{E--Mail}} $|$ \href{http://www.cbs.ac.in/people/physics-faculty/manojendu-choudhury}{\textcolor{blue}{Webpage}} \\
%\end{tabular}

\end{resume}
\end{document}