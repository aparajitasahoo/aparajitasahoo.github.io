\documentclass[margin,line]{res}

\usepackage{hyperref}
\usepackage{amsmath}
\usepackage{textcomp}
\usepackage{color}
\usepackage{lettrine}
%\usepackage[left=0.75in, right=0.75in, top=0.75in,bottom=0.75in, margin=0.75in]{geometry}
\oddsidemargin = -.5in
\evensidemargin = -.5in
\topmargin = -0.3in
\textheight = 9.75in

\textwidth = 6.0in
\itemsep=0in
\parsep=0in
% if using pdflatex:
%\setlength{\pdfpagewidth}{\paperwidth}
%\setlength{\pdfpageheight}{\paperheight}

\newenvironment{list1}{
  \begin{list}{\ding{113}}{%
      \setlength{\itemsep}{0in}
      \setlength{\parsep}{0in} \setlength{\parskip}{0in}
      \setlength{\topsep}{0in} \setlength{\partopsep}{0in}
      \setlength{\leftmargin}{0.17in}}}{\end{list}}
\newenvironment{list2}{
  \begin{list}{$\bullet$}{%
      \setlength{\itemsep}{0in}
      \setlength{\parsep}{0in} \setlength{\parskip}{0in}
      \setlength{\topsep}{0in} \setlength{\partopsep}{0in}
      \setlength{\leftmargin}{0.2in}}}{\end{list}}

\tolerance=1
\emergencystretch=\maxdimen
\hyphenpenalty=10000
\hbadness=10000

\pagenumbering{gobble}

\begin{document}

%\name{\textbf{\huge{Alankar Kotwal}} \vspace*{.05in} \newline  {\sc Senior Undergraduate}\vspace*{.1in}}
\name{\textbf{\huge{Aparajita Sahoo}} \vspace*{.1in} }

\begin{resume}
\section{\sc Contact Information}
\vspace{.05in}
\begin{tabular}{@{}p{2.9in}p{.5in}p{3in}}
\href{https://redxmumbai.com/}{\textcolor{blue} {REDX Innovation Lab, Mumbai}} & \multicolumn{1}{r}{\it Phone:}  &(+91) 907 621 1711 \\
\href{http://redx.io/}{\textcolor{blue} {Rethinking Engineering Design eXecution}} &\multicolumn{1}{r}{\it E--Mail:}& \href{mailto:aparajitasahoois@gmail.com}{\textcolor{blue}{aparajitasahoois@gmail.com}} \\
Welingkar Building, Lakhamsi Nappu Road & & \href{mailto:aparajita.redx@gmail.com}{\textcolor{blue}{aparajita.redx@gmail.com}} \\
Matunga (East), Mumbai, India 400 019 & \multicolumn{1}{r}{\it Webpage:} &\href{http://www.aparajitasahoo.com/}{\textcolor{blue}{www.aparajitasahoo.com}} \\
\end{tabular}

\section{\sc Research Interests}
\lettrine[lines=2]{I}{} am passionate about creating a social impact by working at the intersection of health technologies, data science, design and social innovation that will impact the lives of the people globally. I am also interested in enhancing  user experiences by combining the power of computing, mobile technologies and effective design. I have worked on developing low cost healthcare devices which can be deployed in rural areas that lack access to proper medical facilities.

\section{\sc Education}
{\bf \href{http://mu.ac.in/portal/}{\textcolor{blue}{Univ. of Mumbai}}, \href{http://fcrit.ac.in/}{\textcolor{blue}{Fr. Conceicao Rodrigues Inst. of Technology}}}, Mumbai \hfill {\textit{2011 -- 2015}} \\
\vspace*{-.15in}
\begin{list1}
\item[] Bachelor of Engineering, \href{http://www.extc.fcrit.ac.in/}{\textcolor{blue}{Electronics and Telecommunication Engineering}}, \textit{First Class}
\end{list1}

\vspace*{-0.1in}

{\bf \href{http://www.aejcmumbai.ac.in/}{\textcolor{blue}{Atomic Energy Junior College}}}, Anushaktinagar, Mumbai \hfill {\textit{2008 -- 2010}} \\
\vspace*{-.15in}
\begin{list1}
\item[] Intermediate Degree, Science with Vocational Training in Electrical Maintenance, \textit{88\%}
\end{list1}

\vspace*{-0.1in}

{\bf Atomic Energy Central School}, Anushaktinagar, Mumbai \hfill {\textit{1998 -- 2008}} \\
\vspace*{-.15in}
\begin{list1}
\item[] Matriculation, \textit{93.8\%}, \textit{All India Topper in Mathematics}
\end{list1}

%\section{\sc Publications \\ and Preprints}
\section{\sc Conference \\ Presentations}
\begin{list2}
\item Presented the study {\em Fluorescence spectroscopy: A non-invasive technique for clinical diagnosis of skin diseases - A preliminary study } in Asian Dermatological Congress 2016.  %\href{http://biomedicalimaging.org/2017/}{\textcolor{blue} {14$^\text{th}$ International Symposium on Biomedical Imaging (2017)}}. Paper \href{http://alankarkotwal.github.io/pubs/isbi17.pdf}{\textcolor{blue} {here}}.%
\item Presented poster on  {\em Fluorescence spectroscopy: A non invasive technique for clinical diagnosis of skin diseases - A preliminary study } in \href{http://www.dermacon2017kolkata.com/}{\textcolor{blue}{Dermacon 2017.}}
\item Presented project {\em Dermato } in \href{http://mitemergingworlds.com/}{\textcolor{blue}{MIT Emerging Worlds Conference}} in  July 2016 and January 2017.
%Submitted to the \href{http://www.ieee-icassp2017.org/}{\textcolor{blue} {42$^\text{nd}$ International Conference on Acoust., Speech and Signal Processing (2017)}}. Paper \href{http://alankarkotwal.github.io/pubs/icassp17.pdf}{\textcolor{blue} {here}},
%Preprint: \href{https://arxiv.org/abs/1609.02135}{\textcolor{blue} {arXiv:1609.02135 [cs.CV]}}.
%\item -	Oral presentation of  {\em Joint Desmoking and Denoising of Laparoscopy Images} (oral), Proc. of the \href{http://biomedicalimaging.org/2016/}{\textcolor{blue} {13$^\text{th}$ International Symposium on Biomedical Imaging (2016)}}. Paper \href{http://alankarkotwal.github.io/pubs/isbi16.pdf}{\textcolor{blue} {here}}.
%\item Clarke, J. {\em et al.}, {\em Field Robotics, Astrobiology and Mars Analogue Research on the Arkaroola Mars Robot Challenge}, Proc. of the \href{http://www.nssa.com.au/14asrc/14ASRC-proceedings.zip}{\textcolor{blue} {14$^\text{th}$ Australian Space Research Conference 2014}}. Paper \href{http://alankarkotwal.github.io/pubs/asrc14.pdf}{\textcolor{blue} {here}}.
\end{list2}

\vspace*{-0.1in}

\section{\sc Research Experience \\ and Projects}

{\bf  \href{http://redxmumbai.com/}{\textcolor{blue}{REDX Innovation Lab}}}, under {\bf \href{http://mitemergingworlds.com/}{\textcolor{blue}{Emerging Worlds Special Interest Group}}, \href{https://www.media.mit.edu}{\textcolor{blue} {MIT Media Lab}}} \\
{\em Innovation Engineer}, {\em focusing on healthcare technologies for rural areas} \\
\vspace*{-.05in}
\begin{list1}
\item[]\textbf{Project: \href{http://dermato.io/}{\textcolor{blue} {Dermato}}} \hfill $\|$ \hfill {\em Guide: \href{http://www.mit.edu/~ajdas}{\textcolor{blue} {Dr. Anshuman J. Das}}} \hfill $\|$ \hfill {\textit{Oct '15 -- Present}} \\
\textit{A project aiming to solve important problems faced by dermatologists in diagnosis in rural areas}
\vspace{4pt}

\noindent \textbf{The first part} is a study aimed at reducing the subjectivity in the visual examination of skin diseases, especially in depigmented conditions. Here, it is difficult to diagnose similar-looking diseases without an invasive procedure like the biopsy. We have developed a non-invasive method to identify the condition which is much faster, affordable and less painful than existing methods. The device is a fluorescence spectroscopy test that can be done with the help of a mobile phone. I have collaborated with doctors to design and implement this study at the K. J. Somaiya Hospital, Mumbai. I have been actively helping doctors to collect data using our device, from around 260 patients over a period of 5 months. I have worked on analyzing the collected spectral information of various skin diseases and developed a data-driven algorithm to identify patients suffering from vitiligo. We plan to publish the results of this study in a prominent healthcare journal.
\vspace{4pt}

\noindent \textbf{The second part} of the project is a portable mobile clip-on that can be used by dermatologists in the visual examination of the skin. This affordable app-based device provides magnified, high quality and polarized images of skin lesions and allows one to annotate, store and access them along with patient data. Along with giving inputs for its design and usability, I was also involved in a pilot study that included doctors from K. J. Somaiya hospital to test the device on their patients and give us critical feedback on its design and app interface. Its success led us to mass-manufacture the clip-on, to be distributed to about 500 dermatologists in and around Mumbai. \\
\vspace*{-.05in}

\item[]\textbf{Project: StethoCG} \hfill $\|$ \hfill {\em Guide: \href{http://web.media.mit.edu/~guysatat/}{\textcolor{blue} {Guy Satat}}} \hfill $\|$ \hfill {\textit{Jun '15 -- Oct '15}} \\
\textit{A project aiming to provide early  detection of cardiovascular disorders in rural areas}
\vspace{4pt}

\noindent Cardiovascular diseases, among all reasons, cause the maximum number ($\sim$ 1.4 million) of deaths per year in India. Rural areas are especially affected due to a lack of doctors specializing in the diagnosis of these disorders. We aimed to solve this problem with a screening device that helps in detecting heart murmurs in patients leading to possible further diagnosis and treatment. The device combines a digital stethoscope and an ECG to identify irregularities in a heartbeat. I had developed the hardware prototype which led to better data collection. The user interface of the device is designed to be simple so that it can be used by a health worker and its form factor was reduced to make it portable. This was used to collect data samples from various patients and the preliminary data was analyzed to detect similarities in their pattern.\\
This project was presented to Ratan Tata and it got positive feedback.\\
\vspace*{-.05in}

\item[]\textbf{Project: Pediatric Perimeter} \hfill $\|$ \hfill {\em Guide: \href{http://www.lvpei.org/our-team/our-team-Premnandhini.php}{\textcolor{blue} {Dr. Premnandhini Satgunam}}} \hfill $\|$ \hfill {\textit{Jun '13 -- Oct '13}} \\
\textit{A project aiming to test for visual field in infants with cerebral palsy}
\vspace{4pt}

\noindent Peripheral vision impairment is an indicator of potential abnormalities in the human body. It is a challenge to measure the visual field of an infant suffering from physical disabilities like cerebral palsy where the body's movement is restricted and the patient cannot be tested in the standard way. Pediatric Perimeter is a novel device to measure and quantify visual fields and reaction times to light stimulus in infants. This device assists in the early detection of neonatal eye diseases and early signs of vision-threatening conditions.
The prototype we made recorded the response and the movement of the pupil. I was a part of the initial team and was involved in hardware prototyping and design of the enclosure.

\noindent The first version was presented to Dr. A. P. J. Abdul Kalam and received encouraging reviews.
\end{list1}

\section{\sc Internships}

{\bf LVP -- MITRA Fellow, \href{http://lvpmitra.com/}{\textcolor{blue}{Srujana -- Centre of Innovation}}} \hfill  \\
{\em Guides: \href{https://www.media.mit.edu/~sssinha}{\textcolor{blue}{Shantanu Sinha}}, \href{https://web.media.mit.edu/~tswedish/}{\textcolor{blue}{Tristan Swedish}}, \href{https://www.linkedin.com/in/derbedhruv}{\textcolor{blue}{Dhruv Joshi}}} \hfill {\textit{Dec '14}} \\
\vspace*{-.13in}
\begin{list1}
\item[]
I worked on the hardware prototyping of a modular eye diagnostic device that could be used for tracking the movements of pupil. Its design was inspired by the Google Cardboard.
\end{list1}

{\bf Health Tech Intern, \href{https://redxmumbai.wordpress.com/about/}{\textcolor{blue} {MIT Rethinking Diagnostics}}, IIT Bombay} \\
{\em Guide: \href{http://web.media.mit.edu/~guysatat/}{\textcolor{blue}{Guy Satat}}} \hfill {\textit{Jun '14}} \\
\vspace*{-.13in}
\begin{list1}
\item[]
I worked on developing the initial prototype of StethoCG and tested the initial circuits. The design and form factor was made small so that it could be carried easily by a physician. I took up this project again when I joined the REDX Innovation Lab, Mumbai for further development in its design and data collecting capabilities.
\end{list1}

{\bf Product Development Intern, \href{www.qyuki.com}{\textcolor{blue}{Qyuki}}, Mumbai} \\
{\em Guides: \href{http://www.kshitijmarwah.com}{\textcolor{blue}{Kshitij Marwah}} and \href{https://www.linkedin.com/in/ananddhople}{\textcolor{blue}{Anand Dhople}}} \hfill {\textit{Sep '13 -- Nov '13}} \\
\vspace*{-.13in}
\begin{list1}
\item[]
Qyuki, a startup by Oscar winner A. R. Rahman, film maker Shekhar Kapur and Samir Bangara, is a network that brings together artists to collaborate and create content. I worked on its product detailing and wireframing of the website to help in connecting the creators. I was also involved in designing and testing the website interface. This got me interested in designing user interfaces.
\end{list1}

{\bf Industrial Trainee, \href{http://www.barc.gov.in/}{\textcolor{blue}{Bhabha Atomic Research Centre}}, Mumbai} \\
{\em Guide: Dr. Kaloll Roy \hfill {\textit{Jun '13}} }\\
\vspace*{-.13in}
\begin{list1}
\item[]
I worked in the Instrument Maintenance Section of the Research Reactor and Maintenance Division unit. As an Industrial Trainee, I learnt about various electronic instruments used in the plant maintenance and operation. I learnt about many plant equipments and systems.
\end{list1}



\section{\sc Leadership and Initiatives}

%{\bf Founding Members of REDX Innovation Lab} \hfill {\textit{May '15}} \\

{\bf Founding members of \href{https://redxmumbai.com/}{\textcolor{blue} {REDX Innovation Lab, Mumbai}}} \\
\vspace*{-.15in}
\begin{list1}
\item[]
I was among the first members hired for the lab by the Camera Culture group at the MIT Media Lab. This experience of setting up an Innovation Lab from scratch, being involved in critical decisions that would impact the way our lab functioned, managing technical and administrative issues, building an active community of innovators, doctors and corporates around us, collaborating with hospitals to implement our research studies, organizing monthly meetups, building a network of manufacturers and suppliers and designing our products for deployment has been an incredible journey for me. I have learnt a lot about the health innovation ecosystem by interacting regularly with start-ups, clinicians and patients.
\end{list1}


{\bf Organiser of monthly meetups \href{https://www.meetup.com/REDX-Health-Tech-Cafe-Mumbai/}{\textcolor{blue} {REDX: Health Tech Cafe}}, in Mumbai} \\
\vspace*{-.13in}
\begin{list1}
\item[]
Organising monthly REDX Health Tech Cafe meetups which include stakeholders from industry, startups, makerspaces, innovation labs, academia, hospitals and health care  management.
\end{list1}

{\bf Starting \href{http://makersasylum.com}{\textcolor{blue} {Maker's Asylum}}, a Community Makerspace in Mumbai} \hfill {\textit{Nov  '13}} \\
\vspace*{-.13in}
\begin{list1}
\item[]
I was a part of a group of DIY enthusiasts in Mumbai and our passion for tinkering led us to start our own makerspace called Maker's Asylum which kept growing to become one the best community makerspaces in India.
\end{list1}

{\bf Director Of Operations: IEEE -- CRIT Students Chapter } \hfill {\textit{Jun '13 -- Apr '14}} \\
\vspace*{-.13in}
\begin{list1}
\item[]
I held this position in my pre final year where I was actively involved in getting new members to join our IEEE Students Chapter. I was involved in interacting with junior students. I also helped in organizing IEEE competitions and technical festivals.
\end{list1}

{\bf Member of the IETE-ETSA FCRIT Student Council} \hfill {\textit{Jun '12 -- Apr '13}} \\
\vspace*{-.13in}
\begin{list1}
\item[]
It is the intra-departmental student council that manages the departmental fests and competitions. I was a co-opted member of this council and actively took part in organizing and managing various events during my tenure.
\end{list1}

\section{\sc Other \\Projects}
{\bf \href{http://www.roboconindia.com}{\textcolor{blue} {Robocon}}, India} \hfill {\textit{Mar '14}} \\
\vspace*{-.15in}
\begin{list1}
\item[]
Robocon is a national robotics contest where teams from different colleges in India compete to complete various tasks. I was a part of the 30 member team representing my college. I was a part of the electronics team that designed the main circuit board that controlled the motors and powered the robot.
\end{list1}

\vspace*{-0.1in}

{\bf Intelligent Traffic Light Control using Image Processing} \hfill {\textit{Jun '14 to Feb '15}} \\
\vspace*{-.15in}
\begin{list1}
\item[]This project was aimed to develop a real time traffic monitoring system that would dynamically change the duration of traffic lights on roads based on the amount of congestion. I developed the hardware prototype and the algorithm that enabled the dynamic time allocation through lane prioritization using images of the roads taken periodically.
Published paper \href{http://iraj.in/up_proc/pdf/129-14265951081-3.pdf}{\textcolor{blue}{here}}
\end{list1}

\vspace*{-0.1in}

{\bf Manzil} \hfill {\textit{Jan '13}} \\
\vspace*{-.15in}
\begin{list1}
\item[]
Manzil is a landmark based navigation system that can solve navigation issues at a very local level. It works on the basis of voice commands to generate a landmark based map of the locality that guides the user to reach the desired location.
\end{list1}

\section{\sc Talks and \\ Participations}
{Gave a talk on Innovation in Lotus College of Optometry} \hfill {\textit{May '16}}

\vspace*{-0.1in}

{Presented at Maker Mela, India's largest Maker Movement, Mumbai} \hfill {\textit{Oct '15}}

\vspace*{-0.1in}

{Moderated the Education and Skills Development track in Smart City Ideation Initiative,\\ Nashik, India} \hfill {\textit{Sep '15}}

\vspace*{-0.1in}

{Participated in Maker Fest held in National Institute of Design, Ahmedabad, India} \hfill {\textit{Jan '14}}

\section{\sc Key \\Coursework}
\begin{tabular}{ccccc}
Computer Programming & $|$ & Signal Processing & $|$ & Computer Networks \\
Machine Learning (Coursera) & $|$ & Digital Image Processing & $|$ & Wireless Networks \\
Java & $|$ & Interaction Design (Coursera) & $|$ & Mobile Communication
\end{tabular}

\section{\sc Technical \\Skills}
\begin{tabular}{@{}p{1.3in}p{4.3in}}
\textbf{Software} & Matlab, R, Python, C, C++, Java, Solidworks, \LaTeX \\
\vspace*{-0.06in}
\textbf{Hardware} &
\vspace*{-0.06in}
Arduino, Teensy, Olimex, IoT development boards \\
\end{tabular}

\section{\sc Other \\Interests}
\lettrine[lines=2]{I}{} like reading books, listening to music and watching movies. I like doodling in my spare time. I am a foodie and like travelling.

\section{\sc References}
\begin{tabular}{@{}p{3in}p{3in}}
\textbf{Guy Satat}, Camera Culture & \textbf{Shantanu Sinha}, Camera Culture \\
MIT Media Lab $|$ \href{mailto:guysatat@mit.edu}{\textcolor{blue}{E--Mail}} $|$ \href{http://web.media.mit.edu/~guysatat/}{\textcolor{blue}{Webpage}} & Meta Vision $|$ \href{mailto:s.sinha@metavision.com}{\textcolor{blue}{E--Mail}} $|$ \href{https://www.media.mit.edu/~sssinha}{\textcolor{blue}{Webpage}} \\
\end{tabular}
\vspace{-0.15in}

\begin{tabular}{@{}p{3in}p{3in}}
\textbf{Anshuman Das}, Camera Culture & \\ %\textbf{Ashutosh Richhariya}, Ophthalmic Biophysics \\
MIT Media Lab $|$ \href{mailto:ajdas@mit.edu}{\textcolor{blue}{E--Mail}} $|$ \href{http://ajdas.scripts.mit.edu/ajdas/}{\textcolor{blue}{Webpage}} & \\ %LVPEI $|$ \href{mailto:ashutosh@lvpei.org}{\textcolor{blue}{E--Mail}} $|$ \href{http://www.lvpei.org/our-team/our-team-ashutosh.php}{\textcolor{blue}{Webpage}} \\
\end{tabular}
%\vspace{-0.15in}
%
%\begin{tabular}{@{}p{3in}p{3in}}
%\textbf{Prof. Mayank Vahia}, Astrophysics & \textbf{Dr. Aniket Sule}, Astronomy \\
%TIFR $|$ \href{mailto:vahia@tifr.res.in}{\textcolor{blue}{E--Mail}} $|$ \href{http://www.tifr.res.in/~vahia/}{\textcolor{blue}{Webpage}} & HBCSE--TIFR $|$ \href{mailto:anikets@hbcse.tifr.res.in}{\textcolor{blue}{E--Mail}} $|$ \href{http://www.hbcse.tifr.res.in/people/academic/aniket-sule}{\textcolor{blue}{Webpage}} \\
%\end{tabular}
%\vspace{-0.15in}
%
%\begin{tabular}{@{}p{3in}p{3in}}
%\textbf{Prof. Rajbabu Velmurugan}, EE & \textbf{Dr. Manojendu Choudhury}, Astrophysics \\
%IITB $|$ \href{mailto:rajbabu@ee.iitb.ac.in}{\textcolor{blue}{E--Mail}} $|$ \href{https://www.ee.iitb.ac.in/web/faculty/homepage/rajbabu}{\textcolor{blue}{Webpage}} & UM--DAE CBS $|$ \href{mailto:manojendu@cbs.ac.in}{\textcolor{blue}{E--Mail}} $|$ \href{http://www.cbs.ac.in/people/physics-faculty/manojendu-choudhury}{\textcolor{blue}{Webpage}} \\
%\end{tabular}
%\vspace{-0.15in}

%\begin{tabular}{@{}p{3in}p{3in}}
%\textbf{Prof. Rajbabu Velmurugan}, EE & \textbf{Dr. Manojendu Choudhury}, Astrophysics \\
%IITB $|$ \href{mailto:rajbabu@ee.iitb.ac.in}{\textcolor{blue}{E--Mail}} $|$ \href{https://www.ee.iitb.ac.in/web/faculty/homepage/rajbabu}{\textcolor{blue}{Webpage}} & UM--DAE CBS $|$ \href{mailto:manojendu@cbs.ac.in}{\textcolor{blue}{E--Mail}} $|$ \href{http://www.cbs.ac.in/people/physics-faculty/manojendu-choudhury}{\textcolor{blue}{Webpage}} \\
%\end{tabular}

\end{resume}
\end{document}
